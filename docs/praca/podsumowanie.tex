\chapter{Podsumowanie}
\thispagestyle{chapterBeginStyle}

Udało się sprostać prawie wszystkim wymaganiom funkcjonalnym systemu. Zakończono prace implementacyjne nad pierwszą wersją zarówno aplikacji mobilnej, jak i części serwerowej. System zapewnia skuteczność na wysokim poziomie, choć nadal możliwym do poprawy, tak by jego użycie w trakcie treningu lub zawodów darta było rzeczywistym ułatwieniem procesu obliczania punktów. Pozostaje dużo funkcjonalności i usprawnień, których wykonanie w przyszłości zwiększy użyteczność aplikacji:
\begin{itemize}
  \item umożliwienie przeprowadzenia pełnej rozgrywki (np. w grę \textit{501}) z pomocą systemu, a nie jedynie przekazywania informacji o pojedynczym rzucie
  \item rozwój metody znajdowania rzutki na zdjęciu, tak by program był mniej wrażliwy np. na zmianę oświetlenia
  \item zbadanie, na ile kalibracja kamery może poprawić dokładność
  \item przeprowadzenie większej liczby testów w celu ustalenia optymalnych wartości parametrów oraz wykrycia głównych problemów
  \item napisanie testów jednostkowych
  
\end{itemize}

Największym sukcesem pracy jest udane połączenie działań z wielu dziedzin, zarówno matematycznych, jak i czysto informatycznych. Z tego również powodu, w procesie od analizy problemu, przez wykonanie stelaża, aż do implementacji, należało wykazać się dużym przekrojem umiejętności. Znaczącą wartością pracy są również autorskie obliczenia, które okazały się dość skomplikowane, mimo iż oparte są na fundamentalnych prawach geometrii. 