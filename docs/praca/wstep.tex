\chapter{Wstęp}
\thispagestyle{chapterBeginStyle}

Niniejsza praca prezentuje cały proces prowadzący do automatycznego naliczania punktów w grze w darta za pomocą kamer. Obejmuje to zarówno analizę problemu, obliczenia stojące za użytą metodą, jak i projekt oraz implementację systemu. 

Celem pracy jest zaprojektowanie oraz wykonanie systemu o następujących wymaganiach:
\vspace{2mm}

\noindent a) funkcjonalnych:
\begin{itemize}
  \item wykrywanie momentu wbicia rzutki
  \item obliczanie pozycji wbitej rzutki
  \item analiza kilku rzutów bez konieczności wyciągania lotek z tarczy
  \item wyświetlanie informacji dotyczących rzutu w aplikacji mobilnej
  \item zaznaczanie pozycji rzutki na diagramie tarczy
  
\end{itemize}

\noindent b) niefunkcjonalnych:
\begin{itemize}
  \item działanie w czasie rzeczywistym, pozwalające na płynną grę
  \item niski koszt
  \item duże możliwości konfiguracji
\end{itemize}

Należy podkreślić, że system ma duży potencjał i możliwości zastosowania, ponieważ jest niewiele podobnych rozwiązań na rynku. Pojedyncze, istniejące implementacje dla profesjonalistów (np. system \textit{Scolia}\footnote{Adres: \url{https://scoliadarts.com/}}) gwarantują prawie stuprocentową skuteczność, lecz ich cena jest bardzo wysoka, a kod źródłowy i szczegóły budowy urządzenia nie są dostępne.

Praca składa się z pięciu rozdziałów. W rozdziale pierwszym poddano problem analizie, omawiając podstawy gry w darta, sposób reprezentacji rzutu oraz istniejące ograniczenia. Rozdział drugi przedstawia pomysł na rozwiązanie problemu, w pełni opisuje teoretyczny proces uzyskania pozycji rzutki na tarczy. Wytłumaczone są w nim podstawowe pojęcia oraz obliczenia prowadzące do osiągnięcia celu pracy. W rozdziale trzecim zawarte zostały opis urządzeń użytych w systemie, diagramy UML oraz algorytm przetwarzania obrazu. Rozdział czwarty traktuje o szczegółach implementacyjnych: technologiach użytych w projekcie, strukturze projektu i opisie plików źródłowych. Omówiony został tam również, na przykładzie, efekt końcowy działania programu, testy dokładności oraz problemy, które pojawiły się w trakcie implementacji. W rozdziale piątym pokazano, w jaki sposób należy zainstalować potrzebne zależności oraz uruchomić aplikację. Ostatni rozdział podsumowuje pracę, omawiając, co zostało zrealizowane i jakie są możliwe ścieżki rozwoju projektu.